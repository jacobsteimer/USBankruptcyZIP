% Options for packages loaded elsewhere
\PassOptionsToPackage{unicode}{hyperref}
\PassOptionsToPackage{hyphens}{url}
%
\documentclass[
]{article}
\usepackage{amsmath,amssymb}
\usepackage{iftex}
\ifPDFTeX
  \usepackage[T1]{fontenc}
  \usepackage[utf8]{inputenc}
  \usepackage{textcomp} % provide euro and other symbols
\else % if luatex or xetex
  \usepackage{unicode-math} % this also loads fontspec
  \defaultfontfeatures{Scale=MatchLowercase}
  \defaultfontfeatures[\rmfamily]{Ligatures=TeX,Scale=1}
\fi
\usepackage{lmodern}
\ifPDFTeX\else
  % xetex/luatex font selection
\fi
% Use upquote if available, for straight quotes in verbatim environments
\IfFileExists{upquote.sty}{\usepackage{upquote}}{}
\IfFileExists{microtype.sty}{% use microtype if available
  \usepackage[]{microtype}
  \UseMicrotypeSet[protrusion]{basicmath} % disable protrusion for tt fonts
}{}
\makeatletter
\@ifundefined{KOMAClassName}{% if non-KOMA class
  \IfFileExists{parskip.sty}{%
    \usepackage{parskip}
  }{% else
    \setlength{\parindent}{0pt}
    \setlength{\parskip}{6pt plus 2pt minus 1pt}}
}{% if KOMA class
  \KOMAoptions{parskip=half}}
\makeatother
\usepackage{xcolor}
\usepackage[margin=1in]{geometry}
\usepackage{graphicx}
\makeatletter
\def\maxwidth{\ifdim\Gin@nat@width>\linewidth\linewidth\else\Gin@nat@width\fi}
\def\maxheight{\ifdim\Gin@nat@height>\textheight\textheight\else\Gin@nat@height\fi}
\makeatother
% Scale images if necessary, so that they will not overflow the page
% margins by default, and it is still possible to overwrite the defaults
% using explicit options in \includegraphics[width, height, ...]{}
\setkeys{Gin}{width=\maxwidth,height=\maxheight,keepaspectratio}
% Set default figure placement to htbp
\makeatletter
\def\fps@figure{htbp}
\makeatother
\setlength{\emergencystretch}{3em} % prevent overfull lines
\providecommand{\tightlist}{%
  \setlength{\itemsep}{0pt}\setlength{\parskip}{0pt}}
\setcounter{secnumdepth}{-\maxdimen} % remove section numbering
\ifLuaTeX
  \usepackage{selnolig}  % disable illegal ligatures
\fi
\IfFileExists{bookmark.sty}{\usepackage{bookmark}}{\usepackage{hyperref}}
\IfFileExists{xurl.sty}{\usepackage{xurl}}{} % add URL line breaks if available
\urlstyle{same}
\hypersetup{
  pdftitle={Analsyis\_of\_bankruptcies\_by\_ZIP},
  pdfauthor={Jacob Steimer},
  hidelinks,
  pdfcreator={LaTeX via pandoc}}

\title{Analsyis\_of\_bankruptcies\_by\_ZIP}
\author{Jacob Steimer}
\date{2023-07-27}

\begin{document}
\maketitle

\hypertarget{what-causes-high-bankruptcy-rates-in-some-u.s.-zip-codes}{%
\section{What causes high bankruptcy rates in some U.S. ZIP
codes?}\label{what-causes-high-bankruptcy-rates-in-some-u.s.-zip-codes}}

\hypertarget{introduction}{%
\subsubsection{Introduction}\label{introduction}}

During the height of the COVID-19 pandemic, the number of personal
bankruptcy filings dropped significantly across the country. However,
some parts of the country continue to see far more bankruptcies than
others, in a way that doesn't seem clearly correlated with any one
factor, according to data from the Federal Judicial Center. Also,
people's choices of type of bankruptcy --- Chapter 7 or Chapter 13 ---
vary widely.

In this analysis, I explore the relationship between bankruptcy rates in
U.S. ZIP codes and various demographic variables pulled from the U.S.
Census Bureau. Using machine learning techniques, I explore the
importance of the different variables and build models that can predict
bankruptcy rates given Census data.

\hypertarget{data-overview}{%
\subsubsection{Data Overview}\label{data-overview}}

(See lines 1-28 in ``CodeToFollow.R'')

I obtained bankruptcy data from the Federal Judicial Center website and
demographic data from the U.S. Census Bureau. Because the FJC and Census
website links are subject to change, I uploaded the relevant data from
both websites to GitHub and wrote code for its download.

The bankruptcy data includes one row for every bankruptcy filed during
2022. Columns included the type of bankruptcy and the ZIP code connected
with each debtor.

The Census Data included the following data for each ZIP code:

\begin{itemize}
\item
  The Labor Force Participation Rate (the percentage of adults who are
  either working or looking for work)
\item
  The Unemployment Rate (The percentage of the Labor Force that isn't
  working)
\item
  The Homeownership Rate (the percentage of households that own their
  residence)
\item
  The percentage of residents with bachelor's degrees
\item
  The percentage of residents who are Black
\item
  The population and median income in each ZIP
\end{itemize}

I chose most of these factors because they're important economic
characteristics. I chose the percentage of Black residents because prior
research has shown African Americans are more likely to file Chapter 13
than Chapter 7.

\hypertarget{methodsanalysis}{%
\subsubsection{Methods/Analysis}\label{methodsanalysis}}

\hypertarget{data-preparation}{%
\subparagraph{Data Preparation}\label{data-preparation}}

(See lines 29-68 in ``CodeToFollow.R'')

I cleaned the dataset by removing duplicate cases and filtering it to
include only those filed in 2022. I then determined the count of Chapter
13 bankruptcies, count of total bankruptcies and the percentage of
Chapter 13 bankruptcies using ``summarize'' and ``mutate''.
Additionally, I transformed certain variables for better
interpretability. For instance, I divided Median Income by 10,000
because a difference in \$10,000 between ZIPs is more understandable
than a \$1 difference.

I then split the data into a holdout set, a training set and a testing
set. Since the data only included about 18,000 observations, I used a
10\% partition each time, to ensure both the holdout and testing sets
were large enough.

\hypertarget{modeling-with-linear-regression}{%
\subparagraph{Modeling with linear
regression}\label{modeling-with-linear-regression}}

(See lines 69-97 in ``CodeToFollow.R'')

To determine which demographic characteristics correlate most with
elevated bankruptcy rates, I started with linear models. I fit separate
linear models for total bankruptcies per 1,000 residents (fitlm),
Chapter 13 bankruptcies per 1,000 (fitlm3) and the percentage of
bankruptcies that are Chapter 13 (fitlm5). I experimented with various
subsets of variables along the way (fitlm2, fitlm4 and fitlm6).

When fitting models that could predict the sum of Chapter 13
bankruptcies and of total bankruptcies, race, homeownership rate, and
bachelor's degrees proved the strongest predictors. For the percentage
of bankruptcies that are Chapter 13, college education lost most of its
predictive power, and both the unemployment rate and the labor force
participation rate became more important.

\hypertarget{modeling-with-random-forrest}{%
\subparagraph{Modeling with Random
Forrest}\label{modeling-with-random-forrest}}

(see lines 121-144 in code)

Next, I applied the Random Forrest algorithm to predict the same
outcomes as in the linear regression models. The Random Forrest's
variable importance measurements did not highlight the same variables
that jumped out in our linear models. Also, while the linear models
consistently saw large effects caused by similar variables, the Random
Forrest importance rankings were much different from each other,
depending on whether total bankruptcies, Chapter 13 bankruptcies or
percentage of Chapter 13 bankruptcies were being predicted.

For total bankruptcies, it listed the bachelor's degree and race
variables as most important and homeownership rate as least important.
For Chapter 13 bankruptcies, it listed the labor force participation
rate as most important and race as least important. And for the
percentage of bankruptcies that are Chapter 13, it listed homeownership
rate and race as most important and median income and labor force
participation rate as least important.

\hypertarget{results}{%
\subsubsection{Results}\label{results}}

\hypertarget{linear-models-performance}{%
\subparagraph{Linear models
performance}\label{linear-models-performance}}

(see lines 98-120 in code)

My linear models had by far their lowest RMSEs (around .25) when
predicting the percentage of bankruptcies in a given ZIP that were
Chapter 13 (fitlm5 and fitlm6). It struggled the most in predicting the
total number of bankruptcies in a ZIP (fitlm and fitlm2). But even
there, the RMSE was relatively low (around 1.6).

\hypertarget{random-forrest-performance}{%
\subparagraph{Random Forrest
performance}\label{random-forrest-performance}}

(see lines 145-158 in code)

All three of our Random Forrest models performed slightly better --- in
terms of RMSE --- than their linear model counterparts. However, since
the differences were small, we shouldn't necessarily consider the Random
Forrest results more important than the linear model results.

\hypertarget{aside-1-knn-model-and-performance}{%
\paragraph{Aside \#1: KNN model and
performance}\label{aside-1-knn-model-and-performance}}

(see lines 159-165 in code)

I also attempted to use K-nearest neighbors (KNN) for prediction and
analysis, but the results showed higher root mean square errors (RMSE)
compared to linear regression and Random Forrest. Also, I realized that
for the purposes of this analysis, KNN is not the best type of model,
since it makes it hard to interpret the importance of certain variables.
Consequently, I decided to focus on linear regression and Random
Forrest.

\hypertarget{aside-2-using-the-bankruptcy-data-itself-to-predict-chapter-type}{%
\paragraph{Aside \#2: Using the bankruptcy data itself to predict
Chapter
type}\label{aside-2-using-the-bankruptcy-data-itself-to-predict-chapter-type}}

(see lines 166-197 in code)

As I've continued to research the causes of someone filing a Chapter 13
instead of Chapter 7, I decided to analyze this data in one more way.
The bankruptcy data itself contains far more information than Chapter
and ZIP. So, I wanted to analyze which variables within that dataset can
help us accurately predict Chapter choice.

To do this, I pared the data down into include 9 variables that seemed
like they could affect Chapter choice: Total Assets (TOTASSTS), Real
Property (REALPROP), Personal Property (PERSPROP), Total Liabilities
(TOTLBLTS), Secured Claims (SECURED), Unsecured Priority Claims
(UNSECPR), Unsecured Nonpriority Claims UNSECNPR, Total Debt (TOTDBT),
and Current Monthly Income (CNTMNTHI). I then made Chapter choice a
binary variable.

I then chopped the dataset up into four parts: A final holdout set, a
testing set, a training set and a small training set. I created the
small training set because the data has over 388,000 observations.

Using the small training set, I ran a linear regression to find the most
important variables. These were real property, secured debt and current
monthly income.

Then, using just these variables, I ran a linear regression on the
larger training set. The RMSE turned out quite small, which gives me
confidence that these three factors play large roles in the Chapter
someone chooses.

\hypertarget{conclusions}{%
\subsubsection{Conclusions}\label{conclusions}}

This project primarily accomplished three tasks:

\begin{enumerate}
\def\labelenumi{\arabic{enumi}.}
\item
  It demonstrated the ability of multiple machine learning techniques to
  predict bankruptcy rates (and types) using Census data. These
  techniques achieved low RMSEs.
\item
  It proved that the percentage of Black residents in a given ZIP plays
  a major role in the number and type of bankruptcies there. In fact, it
  seems to play a larger role than any of the economic or educational
  characteristics we evaluated. This further proves the previously
  documented role of racial bias in the bankruptcy system. This finding
  informed a news article, which can be found here:
  \url{https://mlk50.com/2023/09/05/shelby-county-ranks-no-1-in-bankruptcy-heres-why/}.
\item
  It also showed that the homeownership rate, college education, and
  labor force participation rate are better predictors of bankruptcy
  than median income or the unemployment rate.
\end{enumerate}

These results have potential implications for policymakers looking to
address systemic racism and researchers interested in understanding why
some parts of the country have much higher bankruptcy rates than others.

\end{document}
